\documentclass[compress]{beamer}

%\usepackage[latin1,utf8]{inputenc}

%\usepackage[utf8]{inputenc}
%\usepackage[utf8]{inputenc}
\usepackage[T1]{fontenc}
\usepackage[english]{babel}

\usetheme{shadow}
\usepackage{multirow}
\usepackage{tikz}
\usepackage{xcolor}
\usepackage{amsmath}
\usepackage{graphicx}
\usepackage{epstopdf}
\usepackage{textpos}
\usepackage{textcomp}
\usepackage{color, colortbl}
\usepackage{fancybox}
\usepackage{boxedminipage}
\usepackage{xcolor}
\usepackage{lipsum}
\usepackage{comment}
\usepackage{rotating}
\usepackage{xcolor}

\usepackage{epsfig}
\usepackage{amsmath,amssymb}
\usepackage{amsthm}
\usepackage{multirow}
\usepackage{graphicx}
\usepackage{boxedminipage}
\usepackage{latexsym}
\usepackage{subfigure}
\usepackage{wrapfig}
\usepackage{algorithm} 
\usepackage{appendixnumberbeamer}

\usepackage{framed}

\usetikzlibrary{shapes}
\usetikzlibrary{arrows}
\usetikzlibrary{automata}
\usepackage{algorithm}
\usepackage[noend]{algorithmic}
\usepackage{pifont}
\usepackage{fancybox}
\usepackage{listings}
\usepackage{changepage}

% Daniele
\usepackage{adjustbox}

\lstset{escapeinside={<@}{@>}}

% needed if you use oopdf2eps script :)
\epstopdfsetup{suffix=}

\beamertemplatenavigationsymbolsempty 

\title[BlueTracer:
a Robust API Tracer for Evasive Malware]{BlueTracer: \\
a Robust API Tracer for Evasive Malware}

\author[Simone Nicchi]
{
    \textbf{\textcolor{sapienza}{Simone Nicchi}\\~\\{\vspace{-0.5em}\em\textmd{Thesis Advisor: Prof. Camil Demetrescu \\ Thesis Co-Advisors: Dr. Daniele Cono D'Elia, Dr. Emilio Coppa} }\\~\\{Master of Science in Engineering in Computer Science}}\vspace{-2em}
}

%\pgfdeclareimage[height=1.6cm]{logo}{Immagini/sapienza}
%\logo{\centering \pgfuseimage{logo}} 

%\institute[Sapienza Universit\`{a} di Roma, Italy]
%{
  %\inst{1}%
  %Sapienza Universit\`{a} di Roma, Italy\\
  %\and
  %\vskip-2mm
  %\inst{2}%
  %Sapienza Universit\`{a} di Roma, Italy\\
%  \texttt{\space crisafulli.giu@gmail.com}
%}


%\date{\today}
\date{July 20, 2018}
\newcommand{\argmax}{\operatornamewithlimits{argmax}\limits}
\definecolor{forest}{RGB}{21,166,64}
\newcommand\mybox[2][]{\tikz[overlay]\node[ draw=forest, fill=blue!20,inner sep=2pt, anchor=text, rectangle, rounded corners=1mm, #1] {#2};\phantom{#2}}

\begin{document}

\definecolor{sapienza}{rgb}{0.50,0.14,0.20}
\definecolor{LightCyan}{rgb}{0.88,1,1}
\definecolor{Gray}{gray}{0.9}

\newcommand\darrow{\fontsize{36pt}{36pt}\selectfont{\begin{turn}{-90}\bf {\textcolor{sapienza}{\ding{224}}}\end{turn}}}
\newcommand\larrow{\fontsize{36pt}{36pt}\selectfont\bf {\textcolor{sapienza}{\ding{224}}}}
\newcommand\blarrow{\fontsize{36pt}{36pt}\selectfont{\begin{turn}{-135}\bf {\textcolor{sapienza}{\ding{224}}}\end{turn}}}

% insert page number
\newcommand*\oldmacro{}
\let\oldmacro\insertshortauthor% save previous definition
\renewcommand*\insertshortauthor{%
  \leftskip=.3cm% before the author could be a plus1fill ...
  \insertframenumber\,/\,\inserttotalframenumber\hfill\oldmacro}

\newcommand\emlarrow{%
        {\bf {\textcolor{sapienza}{\ding{224}}}}%
}

\setbeamercolor*{alerted text}{bg=white,fg=sapienza}
\setbeamercolor*{structure}{bg=white,fg=sapienza}

\setbeamercolor*{palette primary}{use=structure,fg=white,bg=sapienza}
\setbeamercolor*{palette secondary}{use=structure,fg=white}
\setbeamercolor*{palette tertiary}{use=structure,fg=white}
\setbeamercolor*{palette quaternary}{fg=black,bg=white}

\setbeamercolor*{sidebar}{use=structure,bg=structure.fg}
  
\setbeamercolor*{palette sidebar primary}{use=structure}
\setbeamercolor*{palette sidebar secondary}{fg=white}
\setbeamercolor*{palette sidebar tertiary}{use=structure}
\setbeamercolor*{palette sidebar quaternary}{fg=white}

\setbeamercolor*{titlelike}{parent=palette primary}

\setbeamercolor*{separation line}{}
\setbeamercolor*{fine separation line}{}


\setbeamercolor{frametitle}{fg=white,bg=sapienza!100}
\setbeamercolor{frametitle right}{fg=white,bg=sapienza!100}
\setbeamercolor{subsection in head/foot}{bg=sapienza}
\setbeamercolor{title in head/foot}{bg=sapienza}


% insert page number
%\newcommand*\oldmacro{}
%\let\oldmacro\insertshortauthor% save previous definition
%\renewcommand*\insertshortauthor{%
%  \leftskip=.3cm% before the author could be a plus1fill ...
%  \insertframenumber\,/\,\inserttotalframenumber\hfill\oldmacro}

\newcommand{\stt}{\small\tt}
\newcommand{\word}[1]{\fontsize{10.4}{10}\textsf{#1}}        % mention of word
\newcommand{\wordb}[2]{\fontsize{10.4}{10}\textsf{#1}$_#2$}        % mention of word
\newcommand{\wordc}[3]{\fontsize{10.4}{10}\textsf{#1}$_#2^#3$}        % mention of word
\newcommand{\wikipage}[1]{\textsc{#1}}        % mention of wikipage

\newcommand\tab[1][0.5cm]{\hspace*{#1}}

\begin{frame}
 
 \begin{figure}
     \centering
    \vspace{0.5cm}
     \includegraphics[scale=.40]{image/sapienza.png}\\
    
 \end{figure}
 
\titlepage
\end{frame}


\section{Intro.}

\subsection{Malware Analysis}

\begin{frame}
    \frametitle{Malware: an increasingly significant problem }

    \begin{figure}
        \includegraphics[scale=0.55]{image/malware2}
    \end{figure}
    
\end{frame}

\begin{frame}
    \frametitle{Malware Analysis}

    Two main types:
    \medskip
    \begin{itemize}
        \item \textcolor{sapienza}{\textbf{Static Analysis:}}\\
        involves the inspection of the different data and code sections of a binary
        \item \textcolor{sapienza}{\textbf{Dynamic Analysis:}}\\
        the malware sample is executed and the actions it performs on the environment are observed
    \end{itemize}
    \vspace{0.5cm}    
        
         \begin{beamerboxesrounded}[shadow=true]{}
    Dynamic analysis strongly favoured as it allows to dodge \\ code obfuscations and deal with a large number of samples
    \end{beamerboxesrounded}    

    \begin{textblock*}{2cm}(9.0cm,-6.1cm)
   \includegraphics[width=1.2cm]{image/search.png}% use the \includegraphics command here
	\end{textblock*} 

\end{frame}

\subsection{Function Call Monitoring}
\begin{frame}[fragile]
    \frametitle{Function call monitoring}

\begin{itemize}
\item Functions can abstract
implementation details providing
a semantically richer representation of some functionality
\item The abstractions embodied by \textbf{system calls} and \textbf{library calls} \\ can be used to grasp the visible behavior of a malicious sample 
\end{itemize}
\medskip
\textcolor{sapienza}{\textbf{Example}}:\\
\bigskip

\begin{beamerboxesrounded}[shadow=true]{}

\lstset{
    language=C++,
    stringstyle=\color{sapienza}, % string color
    basicstyle=\linespread{1.5}\ttfamily\small,
}

\begin{lstlisting}[xleftmargin=0pt]
RegCreateKey("...\CurrentVersion\Run\monitor") 
CreateDirectory("C:\Windows\utils")
CreateFile("C\Windows\utils\GFypmMVqJQOEQqy.exe")
\end{lstlisting}

\end{beamerboxesrounded} 



\end{frame}

\begin{frame}[fragile]
    \frametitle{Implementation of function call monitoring}
	
	    \begin{beamerboxesrounded}[shadow=true]{API Hooking}
    The interception of function calls provided by dynamically
linked \\ libraries (DLLs)
    \end{beamerboxesrounded}
    \bigskip
	
	
%	Three broad categories:    
    
%    \begin{itemize}
%    \item Binary Rewriting
%    \begin{itemize}
%    \item Call Redirection
%    \item Function Rewriting
%    \end{itemize}
%    \item Virtual Machine Introspection (VMI)
%    \item \textbf{Dynamic Binary Instrumentation (DBI)}
%    \end{itemize}
    
%    \begin{textblock*}{2cm}(8.3cm,-3cm)
%   \includegraphics[width=2cm]{image/hook.png}% use the \includegraphics command here
%	\end{textblock*}

One technique is \textbf{Dynamic Binary Instrumentation (DBI)}.
\begin{itemize}
\item[$\rightarrow$] The behaviour of an application is inspected at run-time, without the need of recompiling it, via the injection of analysis code
\end{itemize}
\smallskip

\begin{block}{}
\begin{lstlisting}[basicstyle=\ttfamily\large,xleftmargin=50pt]
<@\textcolor{red}{record\_before(libcall\_name, arg1)}@> 
retval = libcall(arg1, &arg2)
<@\textcolor{red}{record\_after(retval, *arg2)}@> 
\end{lstlisting}
\end{block}


\end{frame}

\iffalse
\begin{frame}[fragile]
    \frametitle{Dynamic Binary Instrumentation (DBI)}
	
A dynamic binary analysis
technique in which the behaviour of an application is inspected at run-time, without the need of recompiling it, via the
injection of analysis code. 
\bigskip
\begin{block}{}
\begin{lstlisting}[basicstyle=\ttfamily\large,xleftmargin=50pt]
<@\textcolor{red}{record\_before(libcall\_name, arg1)}@> 
retval = libcall(arg1, &arg2)
<@\textcolor{red}{record\_after(retval, *arg2)}@> 
\end{lstlisting}
\end{block}
\bigskip
\textcolor{sapienza}{\textbf{Problems:}}
\begin{enumerate}
\item Existing products have limited logging capabilites
\item API hooking techniques in literature are not coupled with mechanisms to hide their presence from evasive malware
\end{enumerate}

\end{frame}
\fi

\subsection{Evasion}
\begin{frame}
    \frametitle{The threat posed by evasive malware}
	
		    \begin{beamerboxesrounded}[shadow=true]{Evasive malware}
Malware that conceals its harmful behaviour when detecting a
hostile environment, such as a well-known sandbox
solution
    \end{beamerboxesrounded}
    \bigskip
    
    \begin{figure}
    	\vspace{-0.6cm}
    	\hspace*{0.5cm}
        \includegraphics[width=11cm]{image/evasive.pdf}
    \end{figure}
    
\end{frame}

\begin{frame}
    \frametitle{Existing problems}

\section{Problem}	
Two main problems which need to be addressed:
\\\bigskip
\begin{enumerate}
\item Existing API tracing tools have \textbf{limited logging capabilities}
\smallskip
\begin{itemize}
\item[--] The amount of recorded information is too little
\item[--] It is hard to distinguish the calls made directly from the sample's main executable from the ones made by libraries
\end{itemize}
\bigskip
\item Current API hooking techniques are \textbf{easily detectable} and are not coupled with mechanisms to hide their presence
from evasive malware
\end{enumerate}

    
\end{frame}

\section{BlueTracer}

\subsection{Overview}
\begin{frame}
    \frametitle{BlueTracer}
    \setbeamerfont*{itemize/enumerate subbody}{parent=itemize/enumerate body} 

\medskip

\begin{beamerboxesrounded}[shadow=true]{}
\textbf{BlueTracer} is a robust library and system call tracer for Windows programs specialized in evasive malware
\end{beamerboxesrounded}

\medskip

\textcolor{sapienza}{\textbf{The tool possesses a remarkable logging power:}}
\begin{itemize}
\item Access to a source of calls related information is required
\item Logging operations are made challenging by the heterogeneity of Windows libraries used in malware and the lack of well-structured documentation for their prototypes
\begin{itemize}
\item[$\rightarrow$] Integration of reliable external sources \\ (\textbf{Dr. Memory}
and \textbf{CISCO PyREBox})
\end{itemize} 
\end{itemize}

\textcolor{sapienza}{\textbf{Key features:}}
\begin{itemize}
\item Undetected tracing of input parameters, output buffers and return values of over 17 000 system calls and library calls
\item Logging of asynchronous events
\item Allows to trace only calls from the main executable
\end{itemize}    

\end{frame}

\begin{frame}
    \frametitle{BlueTracer (cont'd)}
\setbeamerfont*{itemize/enumerate subbody}{parent=itemize/enumerate body}    

\textcolor{sapienza}{\textbf{Solution to the detection problem:}}
\begin{itemize}
\item Seamless integration with \textbf{BluePill}, a stealthy execution framework based on \textbf{Intel Pin}
%\item Combines reliable external sources of prototypes information
\end{itemize}


%\textcolor{sapienza}{\textbf{Building blocks:}}
%\begin{itemize}
%\item Based on the \textbf{Intel Pin} DBI framework
%\item Integrated with the \textbf{BluePill} stealthy execution framework
%\item Combines reliable external sources of prototypes information
%\end{itemize}

\bigskip
	    \begin{figure}
    	\vspace{-0.3cm}
    	%\hspace*{1cm}
        \includegraphics[width=6cm]{image/BluePill.pdf}
    \end{figure}

\textcolor{sapienza}{\textbf{Faced implementation challenges:}}
\begin{itemize}
\item Making best use of
Intel Pin's capabilities with respect to run-time CPU and memory costs
\item Addressing some inherent limitations of Intel Pin

\end{itemize}

\end{frame}

\iffalse
\subsection{Intel Pin}

\begin{frame}
    \frametitle{Why Intel Pin ?}
    
	Characteristics:
	\begin{itemize}
	\item \textbf{User-friendliness}
	\item \textbf{Portability}
	\item \textbf{Transparency}
	\item \textbf{Efficiency}
	\item \textbf{Robustness}
	\end{itemize}
	\medskip
	\textbf{Analysis routines:} embody the code to be inserted during the application's execution\\
	\smallskip
	\textbf{Instrumentation routines:} determine where the analysis code has to be placed\\
	\bigskip

%	Different analysis and instrumentation granularities
%	\begin{itemize}
%	\item Instruction, trace, routine and image
%	\end{itemize}
    
    \begin{textblock*}{3cm}(6cm,-6.5cm)
   \includegraphics[width=1.6cm]{image/pin.png}% use the \includegraphics command here
	\end{textblock*}
	
	

\end{frame}

\subsection{BluePill}

\begin{frame}
    \frametitle{Integration with BluePill}
	\textbf{BluePill} is a software toolkit which:
	\begin{itemize}
	\item Allows the simulation of a real production environment a specific malware
sample was intended for
	\item Conceals any virtualization artifacts and software setup which might set off
evasion
	\end{itemize}
	\medskip
	
	    \begin{figure}
    	\vspace{-0.3cm}
    	%\hspace*{1cm}
        \includegraphics[width=7.5cm]{image/BluePill.pdf}
    \end{figure}

\end{frame}
\fi

\subsection{Architecture}

\begin{frame}
    \frametitle{BlueTracer's architecture}
    
    \begin{figure}
    	\vspace{-0.5cm}
    	%\hspace*{1cm}
        \includegraphics[width=11cm]{image/BlueTracer.pdf}
    \end{figure}
    
	

\end{frame}

\section{Evaluation}

\subsection{Al-Khaser}
\begin{frame}
    \frametitle{Evaluation with Al-Khaser}
	\textbf{Al-Khaser} is an open-source application which performs common checks employed by malware families to determine if they are being
executed in an analysis environment.
\\\medskip
Checks divided in categories:
\begin{itemize}
\item \textbf{Anti-Debugging}
\item \textbf{Timing-based}
\item \textbf{Human Interaction Detection}
\item \textbf{Anti-Virtualization}
\item \textbf{Anti-Analysis}
\end{itemize}
\medskip
\begin{beamerboxesrounded}[shadow=true]{}
BlueTracer: 
\begin{itemize}
\item remained undetected thanks to its integration with BluePill
\item managed to track all the checks
\end{itemize}
\end{beamerboxesrounded}

    \begin{textblock*}{3cm}(7.5cm,-5.5cm)
   \includegraphics[width=2cm]{image/lights.png}% use the \includegraphics command here
	\end{textblock*}

\end{frame}

\subsection{Evasive malware samples}
\begin{frame}
    \frametitle{Evaluation with evasive malware samples}
    
	Five highly evasive samples collected by Joe Security:    
    
    \begin{table}[h]
\vspace*{-0.1cm}
\begin{center}
\resizebox{9.2cm}{!}{
 \begin{tabular}{||c c c||}
 \hline
 \textbf{ID} & \textbf{MD5} & \textbf{Name} \\ 
 \hline\hline
 1 & 0af4ef5069f47a371a0caf22ae2006a6 & \textit{banker} \\ 
 \hline
 2 & 9437eabf2fe5d32101e3fbf9f6027880 & \textit{dropper} \\
 \hline
 3 & cbdda646a20d95f078393506ecdc0796 & \textit{trojan}\\
 \hline
 4 & cfdd16225e67471f5ef54cab9b3a5558 & Olympic\\
 \hline
 5 & ef694b89ad7addb9a16bb6f26f1efaf7 & CCleaner\\ 
 \hline
\end{tabular}
}
\end{center}
\end{table}
\vspace{-0.1cm}
Evaluation was done manually and is a time-consuming process:
\begin{itemize}
\item Check if logs are congruous with Joe Security reports
\item Process Monitor as ground truth for system activity
\end{itemize}
\medskip
\begin{beamerboxesrounded}[shadow=true]{}
The
logs collected by BlueTracer reveal behaviors consistent with the analysis reports
authored by Joe Security
\end{beamerboxesrounded}	

\end{frame}

\begin{frame}[fragile]
    \frametitle{Example of tracked malevolent action}

\begin{beamerboxesrounded}[shadow=true]{}
Tracing a particular action of a malware instance allows to understand in detail what the sample's intentions are
\end{beamerboxesrounded}
\bigskip    
\textbf{Example}: dropping a malicious executable
\vspace*{0.3cm}
    
    %\begin{figure}
    %	\vspace{-0.5cm}
    	%\hspace*{1cm}
    %    \includegraphics[width=11cm]{image/SampleCall}
    %\end{figure}
	
	\lstset{
    language=C++,
    frame=tb, % draw a frame at the top and bottom of the code block
    tabsize=4, % tab space width
    showstringspaces=false, % don't mark spaces in strings
    numbers=none, % display line numbers on the left
    commentstyle=\color{ao}, % comment color
    %keywordstyle=\color{blue}, % keyword color
    stringstyle=\color{red}, % string color
    basicstyle=\linespread{1.3}\footnotesize\ttfamily,
    basewidth = {.48em},
   escapeinside={<@}{@>}
}

\begin{lstlisting}
~~1116~~ 138 kernel32.dll!CopyFileA
138 	arg 0: c:\Users\Simuset\Desktop\sample1.exe 
(name=lpExistingFileName, type=char*, size=0x1)
138 	arg 1: <@\alert{C:\textbackslash Windows\textbackslash system32\textbackslash $\dagger$\textbackslash ffpb6966.exe}@> 
(name=lpNewFileName, type=char*, size=0x1)
138 	arg 2: 0x0 (name=bFailIfExists, type=(long/int), size=0x4)
138    executed kernel32.dll!CopyFileA =>
138 	retval: 0x1 (name=Return value, type=(long/int), size=0x4)
\end{lstlisting}

\iffalse
\begin{lstlisting}
~~972~~ 1397 kernel32.dll!CreateFileW
1397 	arg 0: <@\alert{C:\textbackslash Windows  GFypmMVqJQOEQqy.exe}@> 
(name=lpFileName, type=wchar_t*, size=0x2)
1397 	arg 1: 0x40000000 (name=dwDesiredAccess, type=DWORD, size=0x4)
1397 	arg 2: 0x1 (name=dwShareMode, type=DWORD, size=0x4)
1397 	arg 3: 0x00000000 (name=lpSecurityAttributes, type=struct*|_SECURITY_ATTRIBUTES, size=0x60)
1397 	arg 4: 0x2 
(name=dwCreationDisposition, type=DWORD, size=0x4)
1397 	arg 5: 0x0 (name=dwFlagsAndAttributes, type=DWORD, size=0x4)
1397 	arg 6: 0x0 (name=hTemplateFile, type=DWORD, size=0x4)
1397    executed kernel32.dll!CreateFileW =>
1397 	retval: 0x128 (name=Return value, type=DWORD, size=0x4))
\end{lstlisting}
\fi

\lstset{
    language=C++,
    frame=tb, % draw a frame at the top and bottom of the code block
    tabsize=4, % tab space width
    showstringspaces=false, % don't mark spaces in strings
    numbers=left, % display line numbers on the left
    commentstyle=\color{ao}, % comment color
    keywordstyle=\color{blue}, % keyword color
    stringstyle=\color{red}, % string color
    basicstyle=\footnotesize\ttfamily,
    basewidth = {.48em}
} 


\end{frame}

\section{Conclusions}

\begin{frame}
    \frametitle{Conclusions}
	
	\textbf{\textcolor{sapienza}{Contribution:}} \\
	\smallskip	
	\tab Design and implementation of \textbf{BlueTracer},
a robust library and system call tracer for Windows programs specialized in evasive
malware.
	\\\bigskip
	\textbf{\textcolor{sapienza}{Future Developments:}} \\
	\begin{itemize}
	\item Automatic methodology for large-scale evaluation 
	\item Improvement of logging capabilities
	\item Usage of log filtering techniques
	\end{itemize}

\end{frame}

\begin{frame}
\begin{center}
{\fontsize{15}{15}\selectfont \textbf{Thank you for your attention!}}
\end{center}
\end{frame}

\end{document}

