\documentclass[compress]{beamer}

%\usepackage[latin1,utf8]{inputenc}

%\usepackage[utf8]{inputenc}
%\usepackage[utf8]{inputenc}
\usepackage[T1]{fontenc}
\usepackage[english]{babel}

\usetheme{shadow}
\usepackage{multirow}
\usepackage{tikz}
\usepackage{xcolor}
\usepackage{amsmath}
\usepackage{graphicx}
\usepackage{epstopdf}
\usepackage{textpos}
\usepackage{textcomp}
\usepackage{color, colortbl}
\usepackage{fancybox}
\usepackage{boxedminipage}
\usepackage{xcolor}
\usepackage{lipsum}
\usepackage{comment}
\usepackage{rotating}
\usepackage{xcolor}

\usepackage{epsfig}
\usepackage{amsmath,amssymb}
\usepackage{amsthm}
\usepackage{multirow}
\usepackage{graphicx}
\usepackage{boxedminipage}
\usepackage{latexsym}
\usepackage{subfigure}
\usepackage{wrapfig}
\usepackage{algorithm} 
\usepackage{appendixnumberbeamer}

\usepackage{framed}

\usetikzlibrary{shapes}
\usetikzlibrary{arrows}
\usetikzlibrary{automata}
\usepackage{algorithm}
\usepackage[noend]{algorithmic}
\usepackage{pifont}
\usepackage{fancybox}
\usepackage{listings}
\usepackage{changepage}

% Daniele
\usepackage{adjustbox}

\lstset{escapeinside={<@}{@>}}

% needed if you use oopdf2eps script :)
\epstopdfsetup{suffix=}

\beamertemplatenavigationsymbolsempty 

\title[BlueTracer:
a Robust API Tracer for Evasive Malware]{BlueTracer: \\
a Robust API Tracer for Evasive Malware}

\author[Simone Nicchi]
{
    \textbf{\textcolor{sapienza}{Simone Nicchi}\\~\\{\vspace{-0.5em}\em\textmd{Thesis Advisor: Prof. Camil Demetrescu \\ Thesis Co-Advisors: Dr. Daniele Cono D'Elia, Dr. Emilio Coppa} }\\~\\{Master of Science in Engineering in Computer Science}}\vspace{-2em}
}

%\pgfdeclareimage[height=1.6cm]{logo}{Immagini/sapienza}
%\logo{\centering \pgfuseimage{logo}} 

%\institute[Sapienza Universit\`{a} di Roma, Italy]
%{
  %\inst{1}%
  %Sapienza Universit\`{a} di Roma, Italy\\
  %\and
  %\vskip-2mm
  %\inst{2}%
  %Sapienza Universit\`{a} di Roma, Italy\\
%  \texttt{\space crisafulli.giu@gmail.com}
%}


%\date{\today}
\date{July 20, 2018}
\newcommand{\argmax}{\operatornamewithlimits{argmax}\limits}
\definecolor{forest}{RGB}{21,166,64}
\newcommand\mybox[2][]{\tikz[overlay]\node[ draw=forest, fill=blue!20,inner sep=2pt, anchor=text, rectangle, rounded corners=1mm, #1] {#2};\phantom{#2}}

\begin{document}

\definecolor{sapienza}{rgb}{0.50,0.14,0.20}
\definecolor{LightCyan}{rgb}{0.88,1,1}
\definecolor{Gray}{gray}{0.9}

\newcommand\darrow{\fontsize{36pt}{36pt}\selectfont{\begin{turn}{-90}\bf {\textcolor{sapienza}{\ding{224}}}\end{turn}}}
\newcommand\larrow{\fontsize{36pt}{36pt}\selectfont\bf {\textcolor{sapienza}{\ding{224}}}}
\newcommand\blarrow{\fontsize{36pt}{36pt}\selectfont{\begin{turn}{-135}\bf {\textcolor{sapienza}{\ding{224}}}\end{turn}}}

% insert page number
\newcommand*\oldmacro{}
\let\oldmacro\insertshortauthor% save previous definition
\renewcommand*\insertshortauthor{%
  \leftskip=.3cm% before the author could be a plus1fill ...
  \insertframenumber\,/\,\inserttotalframenumber\hfill\oldmacro}

\newcommand\emlarrow{%
        {\bf {\textcolor{sapienza}{\ding{224}}}}%
}

\setbeamercolor*{alerted text}{bg=white,fg=sapienza}
\setbeamercolor*{structure}{bg=white,fg=sapienza}

\setbeamercolor*{palette primary}{use=structure,fg=white,bg=sapienza}
\setbeamercolor*{palette secondary}{use=structure,fg=white}
\setbeamercolor*{palette tertiary}{use=structure,fg=white}
\setbeamercolor*{palette quaternary}{fg=black,bg=white}

\setbeamercolor*{sidebar}{use=structure,bg=structure.fg}
  
\setbeamercolor*{palette sidebar primary}{use=structure}
\setbeamercolor*{palette sidebar secondary}{fg=white}
\setbeamercolor*{palette sidebar tertiary}{use=structure}
\setbeamercolor*{palette sidebar quaternary}{fg=white}

\setbeamercolor*{titlelike}{parent=palette primary}

\setbeamercolor*{separation line}{}
\setbeamercolor*{fine separation line}{}


\setbeamercolor{frametitle}{fg=white,bg=sapienza!100}
\setbeamercolor{frametitle right}{fg=white,bg=sapienza!100}
\setbeamercolor{subsection in head/foot}{bg=sapienza}
\setbeamercolor{title in head/foot}{bg=sapienza}


% insert page number
%\newcommand*\oldmacro{}
%\let\oldmacro\insertshortauthor% save previous definition
%\renewcommand*\insertshortauthor{%
%  \leftskip=.3cm% before the author could be a plus1fill ...
%  \insertframenumber\,/\,\inserttotalframenumber\hfill\oldmacro}

\newcommand{\stt}{\small\tt}
\newcommand{\word}[1]{\fontsize{10.4}{10}\textsf{#1}}        % mention of word
\newcommand{\wordb}[2]{\fontsize{10.4}{10}\textsf{#1}$_#2$}        % mention of word
\newcommand{\wordc}[3]{\fontsize{10.4}{10}\textsf{#1}$_#2^#3$}        % mention of word
\newcommand{\wikipage}[1]{\textsc{#1}}        % mention of wikipage

\newcommand\tab[1][0.5cm]{\hspace*{#1}}

\begin{frame}
 
 \begin{figure}
     \centering
    \vspace{0.5cm}
     \includegraphics[scale=.40]{image/sapienza.png}\\
    
 \end{figure}
 
\titlepage
\end{frame}

\section{Introduction}
\subsection{Malware Analysis}

\begin{frame}
    \frametitle{Malware: an increasingly significant problem }

    \begin{figure}
        \includegraphics[scale=0.55]{image/malware2}
    \end{figure}
    
\end{frame}

\begin{frame}
    \frametitle{Malware Analysis}

    Two main types:
    \medskip
    \begin{itemize}
        \item \textbf{Static Analysis:}\\
        involves the inspection of the different data and code sections of a binary
        \item \textbf{Dynamic Analysis:}\\
        the malware sample is executed and the actions it performs on the environment are observed
    \end{itemize}
    \vspace{1cm}    
        
         \begin{beamerboxesrounded}[shadow=true]{}
    Dynamic analysis strongly favoured as it allows to dodge \\ code obfuscations and deal with a large number of samples
    \end{beamerboxesrounded}    
 

\end{frame}

\subsection{Function Call Monitoring}
\begin{frame}
    \frametitle{Function call monitoring}

Functions can abstract
implementation details providing
a semantically richer representation of some functionality. \\\bigskip
Example:
    \begin{figure}
    	\vspace{-0.4cm}
        \includegraphics[scale=0.6]{image/fcmonitoring}
    \end{figure}

\begin{beamerboxesrounded}[shadow=true]{}
The abstractions embodied by \textbf{system calls} and \textbf{library calls} \\ can be used to grasp the visible behavior of a malicious sample
\end{beamerboxesrounded} 

\end{frame}

\begin{frame}
    \frametitle{Implementation of function call monitoring}
	
	    \begin{beamerboxesrounded}[shadow=true]{API Hooking}
    The interception of function calls provided by dynamically
linked \\ libraries (DLLs)
    \end{beamerboxesrounded}
    \bigskip
	Three broad categories:    
    
    \begin{itemize}
    \item Binary Rewriting
    \begin{itemize}
    \item Call Redirection
    \item Function Rewriting
    \end{itemize}
    \item Virtual Machine Introspection (VMI)
    \item \textbf{Dynamic Binary Instrumentation (DBI)}
    \end{itemize}

\end{frame}

\begin{frame}[fragile]
    \frametitle{Dynamic Binary Instrumentation (DBI)}
	
A dynamic binary analysis
technique in which the behaviour of an application is inspected at run-time via the
injection of analysis code. 
\bigskip
\begin{block}{}
\begin{lstlisting}[basicstyle=\ttfamily\large,xleftmargin=50pt]
<@\textcolor{red}{record(arg1)}@> 
retval = libcall(arg1, &arg2)
<@\textcolor{red}{record(retval, *arg2)}@> 
\end{lstlisting}
\end{block}
\bigskip
\textbf{Problem 1}: existing products have limited logging capabilites

\end{frame}

\subsection{Evasion}

\begin{frame}
    \frametitle{The threat posed by evasive malware}
	
		    \begin{beamerboxesrounded}[shadow=true]{Evasive malware}
Malware that conceals its harmful behaviour when detecting a
hostile environment, such as a well-known sandbox
solution
    \end{beamerboxesrounded}
    \bigskip
    
    \begin{figure}
    	\vspace{-0.2cm}
    	\hspace*{1cm}
        \includegraphics[width=9cm]{image/evasive.pdf}
    \end{figure}
    
    \textbf{Problem 2}: API hooking techniques in literature are not coupled with mechanisms to hide their presence from evasive malware


\end{frame}

\section{BlueTracer}

\begin{frame}
    \frametitle{Our solution: BlueTracer}
	

\end{frame}

\subsection{Intel Pin}

\begin{frame}
    \frametitle{Intel Pin}
	

\end{frame}

\subsection{BluePill}

\begin{frame}
    \frametitle{Integration with BluePill}
	

\end{frame}

\subsection{Architecture}

\begin{frame}
    \frametitle{BlueTracer's architecture}
	

\end{frame}

\section{Evaluation}

\subsection{Al-Khaser}
\begin{frame}
    \frametitle{Al-Khaser}
	

\end{frame}

\begin{frame}
    \frametitle{Sample tracked evasive check}
	

\end{frame}

\subsection{Evasive malware samples}
\begin{frame}
    \frametitle{Evasive malware samples}
	

\end{frame}

\section{Conclusions}

\begin{frame}
    \frametitle{Conclusion and future developments}
	

\end{frame}

\begin{frame}
\begin{center}
{\fontsize{15}{15}\selectfont \textbf{Thank you for your attention!}}
\end{center}
\end{frame}

\end{document}

