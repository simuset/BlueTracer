% !TEX root = LMThesisNicchi.tex

\chapter{Introduction}

\iffalse
Malwares are a threat

Dynamic analysis is favoured over static analysis

One dynamic analysis technique is API hooking/tracing

Malwares are also evasive

Problem: API hooking and tracing evasive samples

Solution BlueTracer
\fi

Malicious software (or malware) is any software specifically designed to bring harm to a computer system. The problem posed by malwares is one which is becoming increasingly important, as new and more sophisticated malwares arise every day and the economical damage for organizations keeps worsening \cite{Cisco}. To face this threat, professionals are typically aided by a range of automatic tools capable of analysing and detecting malicious software.

Malware analysis can be carried out either statically or dynamically. Dynamic analysis encompasses techniques that execute a sample and observe the actions it actually performs, whereas in static analysis the sample is examined without running it. Such techniques have evolved over time to keep track with the increasing complexity and diversity of malwares. However, in recent years, the shift towards automation, caused by the need of dealing with a huge and ever-growing number of samples, together with the rising complexity of obfuscation mechanisms utilized by malwares, has strongly favoured dynamic analysis.  

One of the most employed dynamic analysis techniques is function call monitoring. Generally, a function is made up of code which carries out a particular task, like for example creating a file or printing a message. Although the utilization of functions allows for easy re-usability of code and simpler maintenance, the propriety which makes them particularly valuable from a program analysis perspective is that they abstract the implementation details, providing a semantically richer representation of some functionality. For instance, let us consider a sorting function; it might not be important to know the underlying sorting algorithm as long as it is known that the function sorts the input number set. In the context of dynamic analysis, the abstractions provided by API calls and system calls (or eventually Windows Native APIs) are incredibly helpful since they can be used to grasp the overall behaviour of the sample being analyzed.

The typical technique used for function calls monitoring in dynamic malware analysis is \textit{API hooking}, i.e. the interception of function calls provided by DLLs. The idea is to alter the original sample so that, besides the function of interest, a \textit{hooking} function is also called, which is in charge of performing the wanted analysis, e.g. logging the function invocation on a file or analyzing the function's parameters \cite{Egele:2008:SAD:2089125.2089126}.      

A problem that all dynamic analysis techniques have to face, including function call monitoring by means of API hooking, is the widespread of evasive malwares. Such malwares check whether or not they are being executed in an adverse environment and conceal their harmful behaviours accordingly, like for example by carrying out an exit sequence \cite{BluePill}. Unfortunately, such anti-evasion mechanisms are frequently adopted my malicious samples. According to Symantec's Internet Security Report of March 2018, 18\% of new malware were virtual-machine-aware \cite{Symantec}. To make matters worse, the API hooking techniques presented in literature are easily detectable and are not coupled by any mechanism to hide their presence from evasive malwares.

\paragraph{Contributions.}
Throghout this thesis we present \textbf{BlueTracer}, a robust library and system call tracer for Windows applications, specialized in the monitoring of evasive malwares. BlueTracer is based on the Intel Pin \cite{Pin} dynamic binary instrumentation (DBI) framework and is able to counteract malwares' anti-evasion measures thanks to its integration with BluePill, a software toolkit built on top of a DBI layer which allows the simulation of the execution environment a particular malware was designed for and conceals any virtualization artifacts and setup details which might set off evasion \cite{BluePill}. BlueTracer is capable of tracing the input values, the output values and the return values of an extremely wide range of system calls (including Windows Native APIs) and API calls. Moreover, it also supports the tracing of Windows callbacks functions and Windows asynchronous procedure calls (APC). The tool was tested on a benign application aimed at assessing how good an anti-malware system is against evasion techniques and on actual evasive malwares, proving to be effective in both tracing the samples' activity and remaining undetected.                

\paragraph{Structure of the Thesis.}
The remaining part of this thesis is structured as follows.
Chapter 1 describes the major \textit{API hooking} techniques present in literature, outlining their stengths and weaknesses, especially from a detection point of view. It also introduces the concept of Dynamic Binary Instrumentation (DBI), which constitutes BlueTracer's core.
Chapter 3 focuses on the implementation of the tool, on its structure and the design choices which were made during its development.
Chapter 4 illustrates the experimental results and assesses the tool's effectiveness.
Finally, we present concluding remarks in Chapter 5, together with possible future developments.
