\documentclass[binding=0.6cm,LaM,english,noexaminfo,oneside]{sapthesis} % LaM for a Laurea Magistrale
\usepackage{microtype}
\usepackage{babel}
\usepackage[utf8]{inputenx}
\usepackage{hyperref}
\usepackage{tabularx}

\usepackage[backend=bibtex]{biblatex}
\bibliography{biblio.bib}

\setlength{\parindent}{4em}
\renewcommand{\baselinestretch}{1.5}

\hypersetup{pdftitle={My thesis},pdfauthor={Francesco Biccari}}
\title{BlueTracer: \\ a Robust API Tracer for Evasive Malware}
\author{Simone Nicchi}
\IDnumber{1705157}
\course{Master of Science}
\courseorganizer{Faculty of Information Engineering, Informatics and Statistics}
\AcademicYear{2017/2018}
\copyyear{2018}
\advisor{Prof. Camil Demetrescu}
\coadvisor{Dr. Daniele Cono D'Elia}
\coadvisor{Dr. Emilio Coppa}
\authoremail{nicchi.1705157@studenti.uniroma1.it}
\begin{document}
\frontmatter
\maketitle
\dedication{Ai miei genitori, che non hanno mai smesso di supportarmi}

\begingroup
\clearpage% Manually insert \clearpage
\let\clearpage\relax% Remove \clearpage functionality
\vspace*{-2cm}% Insert needed vertical retraction
\begin{abstract}

\iffalse
Malwares are a threat

Dynamic analysis is favoured over static analysis

One dynamic analysis technique is API hooking/tracing

Malwares are also evasive

Problem: API hooking and tracing evasive samples

Solution BlueTracer
\fi

Malicious software (or malware) is any software specifically designed to bring harm to a computer system. The problem posed by malwares is one which is becoming increasingly important, as new and more sophisticated malwares arise every day and the economical damage for organizations keeps worsening \cite{Cisco}. To face this threat, professionals are typically aided by a range of automatic tools capable of analysing and detecting malicious software.

Malware analysis can be carried out either statically or dynamically. Dynamic analysis encompasses techniques that execute a sample and observe the actions it actually performs, whereas in static analysis the sample is examined without running it. Such techniques have evolved over time to keep track with the increasing complexity and diversity of malwares. However, in recent years, the shift towards automation, caused by the need of dealing with a huge and ever-growing number of samples, together with the rising complexity of obfuscation mechanisms utilized by malwares, has strongly favoured dynamic analysis.  

One of the most employed dynamic analysis techniques is function call monitoring. Generally, a function is made up of code which carries out a particular task, like for example creating a file or printing a message. Although the utilization of functions allows for easy re-usability of code and simpler maintenance, the propriety which makes them particularly valuable from a program analysis perspective is that they abstract the implementation details, providing a semantically richer representation of some functionality. For instance, let us consider a sorting function; it might not be important to know the underlying sorting algorithm as long as it is known that the function sorts the input number set. In the context of dynamic analysis, the abstractions provided by API calls and system calls (or eventually Windows Native APIs) are incredibly helpful since they can be used to grasp the overall behaviour of the sample being analyzed.

The typical technique used for function calls monitoring in dynamic malware analysis is \textit{API hooking}, i.e. the interception of function calls provided by DLLs. The idea is to alter the original sample so that, besides the function of interest, a \textit{hooking} function is also called, which is in charge of performing the wanted analysis, e.g. logging the function invocation on a file or analyzing the function's parameters \cite{Egele:2008:SAD:2089125.2089126}.      

A problem that all dynamic analysis techniques have to face, including function call monitoring by means of API hooking, is the widespread of evasive malwares. Such malwares check whether or not they are being executed in an adverse environment and conceal their harmful behaviours accordingly, like for example by carrying out an exit sequence \cite{BluePill}. Unfortunately, such anti-evasion mechanisms are frequently adopted my malicious samples. According to Symantec's Internet Security Report of March 2018, 18\% of new malware were virtual-machine-aware \cite{Symantec}. To make matters worse, the API hooking techniques presented in literature are easily detectable and are not coupled by any mechanism to hide their presence from evasive malwares.

In this thesis we present \textbf{BlueTracer}, a robust library and system call tracer for Windows applications, specialized in the monitoring of evasive malwares. BlueTracer is based on the Intel Pin \cite{Pin} dynamic binary instrumentation (DBI) framework and is able to counteract malwares' anti-evasion measures thanks to its integration with BluePill, a software toolkit built on top of a DBI layer which allows the simulation of the execution environment a particular malware was designed for and conceals any virtualization artifacts and setup details which might set off evasion \cite{BluePill}. BlueTracer is capable of tracing the input values, the output values and the return values of an extremely wide range of system calls (including Windows Native APIs) and API calls. Moreover, it also supports the tracing of Windows callbacks functions and Windows asynchronous procedure calls (APC). To tool was tested on a benign application aimed at assessing how good an anti-malware system is against evasion techniques and on actual evasive malwares, proving to be effective in both tracing the samples' activity and remaining undetected.                
\\
\textbf{Thesis Structure}. The remaining part of this thesis is structured as follows.
Chapter 1 describes the major \textit{API hooking} techniques present in literature, outlining their stengths and weaknesses, especially from a detection point of view. \\
Chapter 2 introduces the concept of Dynamic Binary Instrumentation (DBI) and presents Intel Pin, the framework used to develop BlueTracer. \\
Chapter 3 focuses on the implementation of the tool, on its structure and the design choices which were made during its development. \\
Chapter 4 illustrates the experimental results and assesses the tool's effectiveness. \\
Finally, in Chapter 5, conclusions are presented, together with possible future developments.     

\end{abstract}
\endgroup

\tableofcontents

\mainmatter
\chapter{API Hooking: State of The Art}
In literature there are many different implementations of API hooking. In this chapter we will provide an outline of the various approaches utilized to hook functions in DLLs, outlining the benefits and the limitations of each technique, with a strong focus on their detection by malicious software. In particular, we will focus on user space API hooking of \texttt{Win32} binaries, since this is BlueTracer's current field of application. Obviously, as it is the norm in malware analysis, we also assume that the program under study is only available in binary form.

Depending on their underlying implementation, API hooking techniques can be divided in three broad categories: \textbf{Binary Rewriting Based}, \textbf{Virtul Machine Introspection (VMI) Based} and \textbf{Dynamic Binary Instrumentation (DBI) based}.   

\iffalse
backmatter turns off chapter numbering
\fi
\begin{section}{Binary Rewriting Based Hooking}

Binary rewriting based hooking involves inserting hooks at the API entries, via one of the following two approaches:
\begin{enumerate}
\item Redirecting all \texttt{call} instructions so that the hook is called instead of the original function.
\item Rewriting the function of interest such that, before its invocation, the hook is executed. 
\end{enumerate} 

In both cases the hook function gains access to all the arguments present on the stack, thus being able to carry out all the required analysis operations.

The main techniques which use the first approach are \textit{Import Address Table (IAT) Patching}, \textit{Export Address Table (EAT) patching} and \textit{Proxy DLL}. On the other hand, the most significant techniques which use the second approach are \textit{inline hooking} and \textit{debugger based hooking} (\textit{Figure 1.1}).



\begin{figure}[h]
\centering
\hspace*{-3em}
\includegraphics[scale=0.8]{Figures/API-hooking-3.pdf}
\caption{\textit{API hooking techniques classification}}
\end{figure}

\subsection{Import Address Table (IAT) patching}

In the header of every Portable Executable (PE) file there is an Import Address Table (IAT) for every dynamic-link library (DLL) that is included by the executable \cite{Berdajs:2010:EAU:1815744.1815746} (\textit{Figure 1.3}). This table is utilized to indicate the location of DLL-imported functions in virtual memory and is filled by the Windows loader with the actual function memory addresses after the executable is loaded in memory.

\begin{figure}[h]
\centering
\hspace*{-3em}
\includegraphics[scale=0.8]{Figures/IAT-3.pdf}
\caption{\textit{IAT in PE header}}
\end{figure}

The idea is to overwrite the original pointer to an imported API function so that, instead of pointing to the original API, it will point to a different function.

\newpage

Despite being extremely simple to implement, IAT patching suffers from a couple of disadvantages, which significantly limit its use in practice:
\begin{itemize}
    \item It is incredibly easy to detect by simply examining the entries of the IAT and checking whether or not each address falls inside the memory range of the DLL that should contain the function \cite{HookingDetection}.
    \item It is ineffective when function pointers are acquired dynamically, e.g. via \texttt{LoadLibrary} and \texttt{GetProcAddress} \cite{Buescher:2011:BIS:2186328.2186347}.
\end{itemize}

\subsection{Export Address Table (EAT) patching}

Export Address Table (EAT) patching is similar to IAT patching, with the difference that DLL export address tables are patched instead. The export address table (EAT) contains the name of every function exported by the DLL together with the relative virtual address (RVA) where the function can be found, which is relative to the DLL base address when loaded in memory. To hook an API function via EAT patching all that is needed is to overwrite the corresponding address in the table with the address of another function.

EAT patching produces similar results to the ones obtained through IAT patching, but, unlike IAT patching, the created hooks are global, i.e. they affect every program which utilizes the altered DLL \cite{Berdajs:2010:EAU:1815744.1815746}.

However, in a similar manner to what occurs for IAT patching, it can be easily detected to by simply examining the entries of the EAT and checking whether or not each RVA, when added to the DLL base address, falls within the DLL memory range \cite{Stuttard:2014:ADC:2616217}.

\subsection{Proxy DLL}

In the Proxy DLL approach to hooking, also known as Trojan DLL, the DLL containing the functions to be hooked is replaced with another one having an identical name and exporting all the symbols of the original DLL \cite{CodeProjectHooking}. In addition to calling the original functions so that they can carry out their tasks, the Proxy DLL may also make available different implementations for the hooked functions \cite{Berdajs:2010:EAU:1815744.1815746}.

Even though a Proxy DLL is trivial to implement, it is also extremely easy to detect since the original DLL is substituted with another file, which is very likely to have a different size. Moreover, checksums could be employed to detect the presence of a Trojan DLL.


\subsection{Inline hooking}

In \textit{inline hooking} the API to be hooked has its initial instructions (at least the first 5 bytes) overwritten with an unconditional jump to a replacement function. In order to ensure that the API's original functionality is not lost due to the modification of its entry point, a \textit{trampoline function} is created, consisting of a copy of the overwritten instructions and an unconditional jump back to the unaltered portion of the original function. As a result of this, the replacement function can invoke the original function by calling the trampoline, after performing all the desired analysis operations \cite{Berdajs:2010:EAU:1815744.1815746}.
\textit{Figure 4} illustrates a program's execution flow before and after the use of \textit{inline hooking}.

\begin{figure}[h]
\centering
\includegraphics[scale=0.8]{Figures/Inline.pdf}
\caption{\textit{(a) Ordinary API call execution flow \\
				  (b) API call with inline hooking}}
\end{figure}

\newpage

\textit{Inline Hooking}, which was made famous by its employment  in the Microsoft \textit{Detours} Windows API hooking library, is one of the most used API hooking techniques since it offers a number of advantages:
\begin{itemize}
\item It is fast and efficient.
\item It can be utilized to hook any code, not just operating systems APIs, but also programmer defined functions \cite{Rootkit}. 
\item Unlike IAT patching, the type of command used to call the function does not matter, meaning that the hooking will be effective regardless of the fact that a function is called using the IAT or using \texttt{LoadLibrary} together with \texttt{GetProcAddress}.
\end{itemize} 

Unfortunately though, \textit{inline hooking} is also affected by some limitations:

\begin{itemize}
\item Can be easily detected, for instance by comparing the code section of system libraries in memory with a matching original copy loaded from the file system to detect library modifications \cite{Buescher:2011:BIS:2186328.2186347} or by searching API entry points for specific patterns (e.g. presence of \texttt{jmp} instructions) \cite{HookingDetection}.
\item It needs additional modifications in the case where the function's entry points includes specific instructions, like ones which contain relative memory addresses. In fact, such instructions cannot be executed from a trampoline as the trampoline is located in a different memory location than the one of the original program code \cite{Berdajs:2010:EAU:1815744.1815746}.
\end{itemize}

\subsection{Debugger Based Hooking}

Hooking through the use of a debugger is realized by instructing the debugger to position a breakpoint at the entry point of the target API function. The placement of a breakpoint involves overwriting the initial instructions of the target API functions with CPU specific instructions, like \texttt{INT 3} for \texttt{IA-32}. These lead the CPU to throw a debug exception in case they are pointed by the current instruction pointer (IP). The exception is then intercepted by the debugger, which is able to deduce the API which is being called by the application from the address at which the exception took place \cite{HookingDetection}. Moreover, the debugger also has total control over the memory contents and the CPU state of the process being debugged.

Contrarily to inline hooking, a debugger can be used to hook functions whose entry points include instructions containing relative addresses \cite{Berdajs:2010:EAU:1815744.1815746}.

On the other hand, a debugger is much easier to detect. In fact, there exist specific Windows APIs whose purpose is to find out whether or not the current process is being debugged. For example, \texttt{IsDebuggerPresent} allows to determine if the calling process is running under a debugger, while \texttt{CheckRemoteDebuggerPresent} checks for the presence of a debugger in a separate process. 
In addition, the \texttt{INT 3} instruction in an API entry point immediately gives away the debugger's presence \cite{HookingDetection}.
\end{section}

\begin{section}{Virtual Machine Introspection (VMI) \\ 
Based Hooking}

Virtual Machine Introspection (VMI) based hooking relies on the idea of executing the target program in an emulated environment, typically with QEMU being used as virtual machine monitor (VMM). Function calls are monitored by comparing the virtual processor's instruction pointer with the RVAs of DLLs' exported functions when added to the DLL base address. Function arguments are also monitored and this is done by providing them to callback routines, which perform the appropriate tracking operations.

In theory, a PC emulator allows to have functionalities similar to the ones of a debugger, i.e. the code being monitored can be stopped at any arbitrary point during its execution, allowing its registers and virtual memory to be inspected, with the added advantage of not being subject to the aforementioned issues related to breakpoints.
Moreover, VMI based hooking is harder to detect with respect to the previously illustrated hooking techniques, since emulation is utilized to execute an unknown binary with a complete operating system in software, without the sample being never ran directly on the processor \cite{Bayer2005TTAnalyzeA}.

The significant drawback of VMI based hooking is that it incurs in the \textit{semantic gap} problem, i.e. the issue of deducing high-level information from the raw system information by making sense of the CPU state and memory contents \cite{Egele:2008:SAD:2089125.2089126}. VMI based hooking tools
might need an in-depth knowledge of kernel data structures or other details at low-level, which could constitute a complication when dealing with proprietary operating systems. For this reason, as of right now, VMI is not as effective in practice as a traditional debugger when investigating a sample.

\end{section}

\begin{section}{Dynamic Binary Instrumentation (DBI) \\
Based Hooking}

Dynamic Binary Instrumentation (DBI) is an analysis technique in which the behavior of a binary application is inspected at run-time via the injection of instrumentation code. Such code, after being injected, executes as a component of the ordinary instruction flow, allowing to learn information about the behavior and the state of a sample at different points during its execution \cite{DBI}. In DBI based hooking, the learnt information refers to which APIs are called and, possibly, with which arguments and return values. 

There indeed exist DBI based API tracing tools that rely on the previously illustrated idea, namely \textit{drstrace} and \textit{drltrace}, which are both built on top of the DynamoRIO \cite{DynamoRIO} DBI framework. In particular, \textit{drstrace} is a system call tracer for Windows, while \textit{drltrace} is an API calls tracer for both Windows and Linux applications. These tools, however suffer from two notable drawbacks:

\begin{itemize}
\item They are not equipped with any mechanism aimed at cloaking the execution environment in order to prevent a malicious sample from detecting the DBI.
\item They are limited in the amount of information recorded relative to the traced APIs. This applies particularly to \textit{drltace}, which, unlike \textit{drstrace}, does not log return values and output values for arguments, in addition to not providing a mechanism for translating enumerations' constants to the appropriate name. Furthermore, both tools do not take into consideration Windows callbacks and asynchronous procedure calls (APC).
\end{itemize}

\end{section}

\begin{section}{Conclusion}

In this chapter we have showed how the state of the art API hooking techniques suffer from a number of remarkable shortcomings, especially when dealing with evasive malwares. In fact, binary rewriting based hooking techniques are all easily detectable, while VMI, although harder to uncover, is affected by the \textit{semantic gap problem}. Finally, existing DBI based API tracing tools are not accompanied by adequate cloaking mechanisms and are limited in the amount of logged information. The aforementioned issues indicate that there is a need for a robust API tracer, specialized in the analysis of evasive malwares and with extensive logging capabilites. This is the rationale at the heart of BlueTracer. 

\end{section}

\cleardoublepage
\phantomsection % Give this command only if hyperref is loaded

\chapter{Dynamic Binary Instrumentation and Intel Pin}

\chapter{BlueTracer}

\chapter{Experimental Results}

\chapter{Conclusions and Future Developments}

\printbibliography 


% Here put the code for the bibliography. You can use BibTeX or
% the BibLaTeX package or the simple environment thebibliography.
\end{document}